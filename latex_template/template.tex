%--------------------------------------------------------------------
%------------------------------ MACROS ------------------------------
%--------------------------------------------------------------------

% Personal Information
\newcommand{\name}{SeongHo Park}
\newcommand{\email}{spark112@hawk.illinoistech.edu}
\newcommand{\submissiondate}{February 13, 2026}

% University and Course Information
\newcommand{\school}{Illinois Institute of Technology}
\newcommand{\department}{Department of Computer Science}
\newcommand{\coursecode}{CS 000}
\newcommand{\coursename}{Course Name}
\newcommand{\assignment}{Homework 1}

% Path to the university logo file (ensure the image has a 1:1 aspect ratio)
\newcommand{\logo}{~/dotfiles/latex_template/logo.png}

%--------------------------------------------------------------------
%---------------------------- FORMATTING ----------------------------
%--------------------------------------------------------------------

% Define document class and global settings
\documentclass[letterpaper,11pt]{article}

% Math libraries
\usepackage{amsmath}
\usepackage{amssymb}
\usepackage{amsfonts}
\usepackage{pgfplots}
\usepackage{circuitikz}
\pgfplotsset{compat=1.17}
\usetikzlibrary{shapes.geometric}

% Page margins
\usepackage[top=1in, bottom=1in, left=1in, right=1in]{geometry}

% Typography
\usepackage[english]{babel}
\usepackage[T1]{fontenc}
\usepackage[utf8]{inputenc}
\usepackage{microtype}

% Set header and footer
\usepackage{fancyhdr}
\setlength{\headheight}{15pt}
\setlength{\headsep}{20pt}
\setlength{\footskip}{25pt}
\pagestyle{fancy}
\fancyhf{}
\fancyhead[L]{\coursecode \space - \assignment}
\fancyhead[R]{\thepage}
\fancyfoot[R]{\rule{\textwidth}{0.5pt}\\\name}

% Paragraph formatting
\setlength{\parindent}{0pt}
\setlength{\parskip}{2pt}
\linespread{1.1}

% Use hyphenation as much as possible, to ensure even word spacing
\tolerance=1
\hyphenpenalty=1

% Clickable and same-style hyperlinks
\usepackage[hidelinks]{hyperref}
\urlstyle{same}

% Set boxed text padding
\setlength{\fboxsep}{7pt}

% Settings for machine-readable and ATS-friendly PDF
\input{glyphtounicode}
\pdfgentounicode=1
\hypersetup{
    pdftitle={\name},
    pdfauthor={\name}
}

% Set up university logo
\usepackage[pages=some]{background}
\usepackage{graphicx}
\backgroundsetup{
  scale=1,
  angle=0,
  opacity=0.2,
  position=current page.north west,
  vshift=-100pt,
  hshift=100pt,
  contents={
    \includegraphics[width=100px,height=100px]{\logo}
    }
}

% Allow multiline math equations to go over pages
\allowdisplaybreaks

%--------------------------------------------------------------------
%------------------------- CUSTOM COMMANDS --------------------------
%--------------------------------------------------------------------

% Problem header
\newcommand{\problem}[1]{
    \vspace{10pt}
    \textbf{\Large Problem #1}\\
    \vspace{-5pt}
}
\newcommand{\subproblem}[1]{
    \vspace{3pt}
    \textbf{\large #1}\\
    \vspace{-3pt}
}

%--------------------------------------------------------------------
%------------------------- DOCUMENT HEADING -------------------------
%--------------------------------------------------------------------

\begin{document}

% Insert university logo
\BgThispage

% Suppress header on the first page
\thispagestyle{empty}

% Title
\begin{center}
    \textsc{\school} \\
    \textsc{\department} \\\vspace{-5pt}
\end{center}
\rule{\textwidth}{1pt} \\\vspace{-15pt}

% Course Information
\begin{center}
    \coursecode: \coursename \\\vspace{10pt}
    \textbf{\LARGE \assignment} \\\vspace{-5pt}
\end{center}
\rule{\textwidth}{2pt} \\\vspace{-5pt}

% Personal Information
\begin{center}
    \large \name \\\vspace{3pt}
    \email \\\vspace{3pt}
    \submissiondate \\\vspace{20pt}
\end{center}

% Set math block spacing for the rest of the document
\setlength{\abovedisplayskip}{0pt}
\setlength{\belowdisplayskip}{0pt}

%--------------------------------------------------------------------
%--------------------------- DOCUMENT BODY --------------------------
%--------------------------------------------------------------------

\problem{1}

\textbf{(a) Checking for Independence:}  
To determine independence, we solve the equation $c_1 \vec{w_1} + c_2 \vec{w_2} + c_3 \vec{w_3} = 0$. This expands to:

\begin{align*}
    c_1 + c_2 + c_3 &= 0 \\
    c_2 + c_3 &= 0 \\
    c_3 &= 0 \\
\end{align*}

From the last equation, we see that $c_3 = 0$. Substituting into the second equation gives $c_2 = 0$, and finally, the first equation gives $c_1 = 0$.  
Thus, the three vectors are \textbf{independent}. \\

\textbf{(b) Orthogonality:}  
The three vectors $\vec{w_k}$ are \textbf{not orthogonal} since $\langle \vec{w_1}, \vec{w_2} \rangle = 1$, $\langle \vec{w_2}\vec{w_3} \rangle = 2$, and $\langle \vec{w_1}, \vec{w_3} \rangle = 1$. 
Hence, we apply the Gram-Schmidt orthogonalization process, obtaining the orthogonal vectors $\vec{b_1}, \vec{b_2}, \vec{b_3}$:

\begin{align*}
    \vec{b_1} &= \vec{w_1} = \begin{bmatrix} 1 \\ 0 \\ 0 \end{bmatrix} \\
    \vec{b_2} &= \vec{w_2} - P_{\vec{b_1}} \vec{w_2} = \begin{bmatrix} 0 \\ 1 \\ 0 \end{bmatrix} \\
    \vec{b_3} &= \vec{w_3} - P_{\vec{b_1}} \vec{w_3} - P_{\vec{b_2}} \vec{w_3} = \begin{bmatrix} 0 \\ 0 \\ 1 \end{bmatrix}
\end{align*}

\problem{2}

\textbf{(a) Checking for Independence:}  
To check independence, solve $c_1 \vec{w_1} + c_2 \vec{w_2} + c_3 \vec{w_3} = 0$, which when expanded gives the following system:

\begin{align*}
    c_1 + 2c_2 &= 0 \\
    c_1 - c_2 + c_3 &= 0 \\
    c_1 - c_2 - c_3 &= 0 \\
\end{align*}

From the first equation, $c_1 = -2c_2$. Substituting into the second and third equations yields $c_2 = 0$, and thus $c_3 = 0$. Finally, $c_1 = 0$.  
Hence, the vectors $\vec{w_k}$ are \textbf{independent}. \\

\textbf{(b) Orthogonality:}  
The vectors $\vec{w_k}$ are \textbf{orthogonal} since:

\begin{align*}
    \langle \vec{w_1}, \vec{w_2} \rangle = 0 \quad 
    \langle \vec{w_2}, \vec{w_3} \rangle = 0 \quad 
    \langle \vec{w_1}, \vec{w_3} \rangle = 0 \\
\end{align*}

\textbf{(c) Orthonormal Basis:}  
The norms of the vectors are:

\begin{align*}
    |\vec{w_1}| = \sqrt{3} \quad
    |\vec{w_2}| = \sqrt{6} \quad
    |\vec{w_3}| = \sqrt{2} \\
\end{align*}

Thus, the orthonormal basis is:

\begin{align*}
    \hat{b_1} &= \frac{\vec{w_1}}{|\vec{w_1}|} = \frac{1}{\sqrt{3}} \begin{bmatrix} 1 \\ 1 \\ 1 \end{bmatrix} \\
    \hat{b_2} &= \frac{\vec{w_2}}{|\vec{w_2}|} = \frac{1}{\sqrt{6}} \begin{bmatrix} 2 \\ -1 \\ -1 \end{bmatrix} \\
    \hat{b_3} &= \frac{\vec{w_3}}{|\vec{w_3}|} = \frac{1}{\sqrt{2}} \begin{bmatrix} 0 \\ 1 \\ -1 \end{bmatrix} \\
\end{align*}

\vspace{10pt}

\textbf{(d) Transformation Matrix:}  
The coordinate transformation matrix from $\hat{e_i}$ to $\hat{b_k}$ is:

\begin{align*}
    M =
    \begin{bmatrix}
        \frac{1}{\sqrt{3}} & \frac{1}{\sqrt{3}} & \frac{1}{\sqrt{3}} \\
        \frac{2}{\sqrt{6}} & -\frac{1}{\sqrt{6}} & -\frac{1}{\sqrt{6}} \\
        0 & \frac{1}{\sqrt{2}} & -\frac{1}{\sqrt{2}}
    \end{bmatrix}
\end{align*}

\end{document}
